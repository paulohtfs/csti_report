\section{Estudo de Caso: o Ministério das Comunicações}

\subsection{Características}

Como cada entidade possui seu próprio processo interno, é esperado que todas
elas tenham peculiaridades em sua organização interna. Antes de propor uma solução
é necessário entender como funciona o processo atual dentro do Ministério para
então adequar o processo às necessidades do próprio Ministério.

Através da reunião com os gestores do Ministério, foi possível inferir as
seguintes características:

\begin{itemize}
  \item As ordens de serviços estão atreladas a um contrato;
  \item Cada contrato tem um orçamento definido e um período de vigência;
  \item Ao final da execução de uma ordem de serviço o valor pago à Prestadora
  é abatido do balanço do contrato.
\end{itemize}

\subsection{Problemas Detectados}

Além das características citadas anteriormente, também existem problemas no
modelo de gerenciamento atual que diminuem a eficiência do Ministério quanto
a gestão das ordens de serviço. Compreender os problemas é essencial para propor
uma solução robusta.

Os principais problemas detectados são:

\begin{itemize}
  \item O controle das ordens de serviço não é integrado à ferramenta de gestão
  empregada no Ministério;
  \item Incertezas sobre o orçamento dos contratos;
  \item Excesso de informação em vários documentos;
\end{itemize}

\begin{abstract}
The Information Access Law has put a lot of pressure in public agencies. Under
the society's vigilance, many problems that have been ignored for years now
have visibility, among them, the IT service management. Due to lack of both
experience and knowledge, many agencies created obscure and permissive biddings,
allowing companies to abuse the contracts in order to earn money easily and
to not deliver good services. With the Normative Instruction 04, a new dawn
arises in the IT contract management because now there are clear instructions
and good practices about what should be done, like the service order
management, which is the main topic of the present work.
\end{abstract}

\begin{resumo}
A Lei de Acesso à Informação colocou forte pressão nos órgãos públicos. Sob a
vigilância da sociedade, muitos problemas que por anos vinham sendo ignorados
passaram a ter visibilidade, dentre eles, o gerenciamento de serviços de TI. Por
falta de experiência e conhecimento, muitos órgãos criavam editais obscuros e
permissivos, permitindo que empresas abusassem dos contratos para ganhar dinheiro
à custa da qualidade do produto final. Com o advento da Instrução Normativa 04,
uma nova era da gestão pública dos contratos de TI nasce pois agora há instruções
e recomendações claras do que deve ser feito. Dentre elas, o gerenciamento de
ordens se serviço, que é o tópico principal deste trabalho.
\end{resumo}

\section{Conceitos Principais}

É importante definir alguns conceitos envolvidos quando se tratando de ordens
de serviço. As fundamentais são:

\begin{itemize}
  \item \textbf{Contratada.} A entidade que inicia o processo de uma ordem de
  serviço a partir de uma necessidade interna justificada;
  \item \textbf{Prestadora.} A entidade que executa a ordem de serviço;
  \item \textbf{Ordem de Serviço.} Documento que formaliza uma ou mais
  necessidades, critérios de avaliação que indicam como detectar se elas foram
  satisfeitas além das aspectos legais envolvidos;
\end{itemize}

\subsection{Ordem de Serviço}

A ordem de serviço própriamente dita é um documento que formaliza necessidades,
o que deve ser feito para satisfazê-las e também os critérios para avaliar as
soluções.

De acordo com a IN4, uma OS deve possuir:

\begin{itemize}
  \item \textbf{Necessidade.} A motivação da ordem de serviço;
  \item \textbf{Justificativa.} A definição dos problemas envolvidos dadas as
  necessidades;
  \item \textbf{Itens a serem desenvolvidos.} Todos os itens que, quando
  desenvolvidos, irão satisfazer as necessdades;
  \item \textbf{Critérios de Avaliação.} Critérios para avaliar os itens
  desenvolvidos pela Prestadora;
  \item \textbf{Envolvidos.} Os responsáveis pela gerência da ordem de serviço;
  \item \textbf{Cronograma.} O cronogama a ser seguido pela Contratada e pela
  Prestadora;
  \item \textbf{Custos.} O custo total para executa a ordem de serviço.
\end{itemize}

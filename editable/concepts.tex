\section{Conceitos Principais}

Dentro das proximas subseções serão apresentados os conceitos principais
levantados para serem considerados na construção do modelo de processo
e do protótipo de OS.

\subsection{Instrução Normativa 4}

Dentro da disposição da IN4, que é dividida em seções I, II e II.
Das definições principais para fins da IN4:
\begin{itemize}
  \item \textbf{Gestor de Contratos.} Servidor com atribuições gerenciais,
  técnicas e operacionais relacionadas ao processo de gestão do contrato;
  \item \textbf{Fiscal Técnico do Contrato.} Servidor representante da Área de
  Tecnologia da Informação para fiscalizar tecnicamente o contrato;
  \item \textbf{Fiscal Administrativo do Contrato.} Servidor representante da
  Área Administrativa para fiscalizar o contrato quanto aos aspectos
  administrativos;
  \item \textbf{Fiscal Requisitante do Contrato.} Servidor representante da
  Área Requisitante da Solução, para fiscalizar o contrato do ponto de vista
  funcional da Solução de Tecnologia da Informação;
  \item \textbf{Preposto.} Funcionário representante da contratada,
  responsável por acompanhar a execução do contrato e atuar como interlocutor
  principal junto à contratante, incumbido de receber, diligenciar, encaminhar
  e responder as principais questões técnicas, legais e administrativas
  referentes ao andamento contratual;
  \item \textbf{Contratante.} A entidade que inicia o processo de uma ordem de
  serviço a partir de uma necessidade interna justificada;
  \item \textbf{Prestadora.} A entidade que executa a ordem de serviço;
  \item \textbf{Ordem de Serviço.} Documento que formaliza uma ou mais
  necessidades, critérios de avaliação que indicam como detectar se elas foram
  satisfeitas além das aspectos legais envolvidos;
\end{itemize}

\subsection{Ordem de Serviço}

A ordem de serviço própriamente dita é um documento que formaliza necessidades,
o que deve ser feito para satisfazê-las e também os critérios para avaliar as
soluções.

De acordo com a IN4, uma OS deve possuir:

\begin{itemize}
  \item \textbf{Necessidade.} A motivação da ordem de serviço;
  \item \textbf{Justificativa.} A definição dos problemas envolvidos dadas as
  necessidades;
  \item \textbf{Itens a serem desenvolvidos.} Todos os itens que, quando
  desenvolvidos, irão satisfazer as necessdades;
  \item \textbf{Critérios de Avaliação.} Critérios para avaliar os itens
  desenvolvidos pela Prestadora;
  \item \textbf{Envolvidos.} Os responsáveis pela gerência da ordem de serviço;
  \item \textbf{Cronograma.} O cronogama a ser seguido pela Contratada e pela
  Prestadora;
  \item \textbf{Custos.} O custo total para executa a ordem de serviço.
\end{itemize}

\section{Introdução}

Em um momento onde a sociedade vem exigindo mais transparência do Estado e onde
a informação é de fácil acesso e barata, vem aumentando cada vez mais a demanda
pela publicação de informações da gestão interna dos órgãos públicos com o
intuito de garantir a todo cidadão o direito de saber como os recursos públicos
estão sendo utilizados.

Com o advento da Lei de Acesso à Informação, os órgãos públicos passaram a ser
pressionados a divulgar seus dados. Esse evento tão logo mostrou uma grande
fragilidade desses órgãos: a ineficiência da gestão. Seja intencional ou por
ingenuidade, o fato é que muitos órgãos públicos não possuíam (e ainda não
possuem) controle sobre seus gastos.

No início dos anos 2000, com uma onda de informatização acontecendo por todo o
Estado, esse problema afetou de forma muito grave as contratações os serviços
de TI. Por falta de experiência dos gestores, além dos editais para a
contratação de serviços de TI serem confusos e abstratos, as empresas recebiam
o pagamento sem a entrega de um produto funcional.

\subsection{Contexto}

A gerência de ordens de serviço é um problema real no cotidiano dos órgãos
públicos, pois boa parte dos gestores responsáveis pela TI não tem formação específica
nas áreas de computação ou possuem formação, mas o seu foco se limita
apenas às área técnica. Além desse problema, há casos onde as dificuldades
dessa gestão se dá pela ausência ou ineficiência das ferramenas empregadas.

O Ministério das Comunicações é um exemplo do segundo caso: as ferramentas
empregadas não contemplam diversas funcionalidades específicas do ministério,
obrigando, por muitas vezes, os gestores a utilizarem ferramentas externas como
planilhas eletrônicas, das quais são impróprias para a gestão desses ativos.

A Instrução Normativa 4 prevê uma série de boas práticas e recomendações para
a contratação de serviços de TI por órgãos públicos, indo desde a definição
dos conceitos básicos de um serviço até à manutenção de contratos. Com essa
norma, os órgãos públicos têm uma referência do que deve ou não estar previsto
em um modelo de contratação de serviços de TI.

Sob esta ótica, o objetivo do presente trabalho é elaborar, de acordo com a
IN4 e restrições legislativas impostas ao Ministério das Comunicações: um
processo de gerência de ordens de serviços(OS) e um protótipo de OS para
realizar o registros para acompanhamento e monitoramento.

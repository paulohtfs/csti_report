\section{Introdução}

Em um momento onde a sociedade vem exijindo mais transparência do Estado e onde
a informação é de fácil acesso e barata, vem aumentando cada vez mais a demanda
pela publicação de informações da gestão interna dos órgãos públicos com o
intuito de garantir a todo cidadão o direito de saber como os recursos públicos
estão sendo utilizados.

Com o advento da Lei de Acesso à Informação, os órgãos públicos passaram a ser
pressionados a divulgar seus dados. Esse evento tão logo mostrou uma grande
fragilidade desses órgãos: a ineficiência da gestão. Seja intencional ou por
ingenuidade, o fato é que muitos órgãos públicos não possuíam (e ainda não
possuem) controle sobre seus gastos.

No início dos anos 2000, com uma onda de informatização acontecendo por todo o
Estado, esse problema afetou de forma muito grave as contratações os serviços
de TI. Por falta de experiência dos gestores, muitos dos editais para a
contratação de serviços de TI eram confusos e abstratos, permitindo que as
empresas pudessem receber sem entregar produtos funcionando.

A Instrução Normativa 4 prevê uma série de boas práticas e recomendações para
a contratação de serviços de TI por órgãos públicos, indo desde a definição
dos conceitos básicos de um serviço até à manutenção de contratos. Com essa
norma os órgãos públicos passaram a ter uma referência do que deve ou não estar
em um contrato.

Um dos itens da IN4, e a que será abordada neste trabalho, é a gerência de
ordens de serviço: dar manutenção em softwares existentes, evoluir
funcionalidades ou mesmo implementar novas soluções.

\subsection{Contexto}

A gerência de ordens de serviço é um problema real no cotidiano dos órgãos
públicos pois boa parte dos gestores responsáveis pela TI não tem formação em
computação ou então possuem formação em alguma área da computação mas têm foco
apenas na área técnica. Além desse problema, há casos onde as dificuldades dessa
gestão se dá pela ausência ou ineficiência das ferramenas empregadas.

O Ministério das Comunicações é um exemplo do segundo caso: as ferramentas lá
empregadas não contemplam diversos casos de uso específicos do ministério,
obrigando, por muitas vezes, os gestores a utilizarem ferramentas externas como
planilhas eletrônicas.

Sob esta ótica, o objetivo do presente trabalho é elaborar, em acordo com a IN4,
um processo para gerênia de ordens de serviço, bem como propor um protótipo de
uma ferramenta que implemene tal processo.
